\section{Software verification report}
% Reference: DS/ISO 26262-6 section 7.4.18
%
% Pre-requisites:
%    - Software safety requirements specification
%    - Software architectural design specification
%
% Reference: DS/ISO 26262-6 section 7.3.1

\subsection{Pre-existing information}
% The following information can be considered:
%    - Technical safety concept
%    - System design specification
%    - Available third party software components
%    - Tool application guidelines
%
% Reference: DS/ISO 26262-6 section 7.3.2

\subsection{SYSTEM NAME}
\subsubsection{Architectural design verification}
% The software architectural design shall be verified to demonstrate the following
% properties:
%    - Compliance with the software safety requirements
%    - Compatibility with the target hardware, including resources
%
% The following verification methods may be used, depending on the assigned ASIL
% of the software component:
%    - Walk-through of the design (or model, if using model-based development)
%    - Inspection of the design (or model, if using model-based development)
%    - Simulation of dynamic parts of the design
%    - Prototype generation
%    - Formal verification
%    - Control flow analysis
%    - Data flow analysis
%
% Required information:
%    -  software design verification of the software architecture of the system
%
% References:
%    - DS/ISO 26262-6 section 7.4.18
%    - Verification methods: DS/ISO 26262-6 Table 6

\subsubsection{Unit design verification}
% Pre-requisites:
%    - Software unit design specification
%    - Software unit implementation
%
% The software unit design and implementation shall be verified to demonstrate:
%    - Fulfillment of the software safety requirement allocated to each unit
%    - Compatibility of the software unit implementation with the target hardware
%
% The following verification methods can be applied, depending on the assigned
% ASIL of the software unit:
%    - Walk-through of the code/model
%    - Inspection of the code/model
%    - Semi-formal verification
%    - Formal verification
%    - Control flow analysis
%    - Data flow analysis
%    - Static code analysis
%    - Semantic code analysis
%
% Required information:
%    -  software design and implementation verification of the software unit
%
% References:
%    - DS/ISO 26262-6 section 8.4.5
%    - Verification methods: DS/ISO 26262-6 Table 9

\subsubsection{Unit testing}
% Pre-requisites:
%    - Software unit design specification
%    - Software unit implementation
%
% The software unit implementation shall be tested to demonstrate that the
% software units achieve:
%    - Compliance with the software unit design specification
%    - The specified functionality
%    - Confidence in the absence of unintended functionality
%    - Robustness
%    - Sufficient resources to support the implementation
%
% The following testing methods can be applied, depending on the assigned
% ASIL of the software unit:
%    - Requirements-based test
%    - Interface test
%    - Fault injection test (e.g. corrupting variable values or introducing code errors)
%    - Resource usage test
%    - Back-to-back comparison between model and code (if possible)
%
% References:
%    - DS/ISO 26262-6 section 9.4.2 -- 9.4.3
%    - Testing methods: DS/ISO 26262-6 Table 10

\subsubsection{Integration testing}
% Pre-requisites:
%    - Software architectural design specification
%    - Software unit implementation
%
% The software units must be integrated and it must be demonstrated that the
% software architectural design is realized by the implementation.
%
% The following testing methods can be applied, depending on the assigned
% ASIL of the software unit:
%    - Requirements-based test
%    - Interface test
%    - Fault injection test (e.g. corrupting variable values or introducing code errors)
%    - Resource usage test
%    - Back-to-back comparison between model and code (if possible)
%
% References:
%    - DS/ISO 26262-6 section 10.4.2 -- 9.4.3
%    - Testing methods: DS/ISO 26262-6 Table 13
%
%
% <<< Repeat these sections for each system required to implement the item >>>
