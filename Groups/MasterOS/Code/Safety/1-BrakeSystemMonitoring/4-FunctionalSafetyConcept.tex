\section{Functional safety concept}
% Reference: DS/ISO 26262-3 section 8
%
% Pre-requisites:
%    - Item definition
%    - Hazard analysis and risk assessment
%    - Safety goals
% 
% Reference: DS/ISO 26262-3 section 8.3.1

\subsection{Pre-existing information}
% The following information can be considered:
%    - Prior architectural assumptions (non-safety related)
%
% Reference: DS/ISO 26262-3 section 8.3.2
The brake monitoring system architecturally consists of the following parts:
\begin{itemize}
\item Brake pedal position sensor
\item Computing system
\item Brake light.
\end{itemize}

\subsection{Derivation of functional safety requirements}
% Functional safety requirements must be specified following DS/ISO 26262-8 section 6.
%
% A functional safety requirement specifies the safety measures and mechanisms that are
% required to comply with the safety goals.
% The functional safety concept is the set of functional safety requirements and the
% architectural elements of the item that they are assigned to.
% The functional safety concept is concerned with:
%    - Fault detection and failure mitigation
%    - Transitioning to a safe state after faults have been detected
%    - Fault tolerance mechanisms, where faults do not lead directly to violation of
%      the safety goals and a safe state is maintained (possibly with degradation)
%    - Fault detection and driver warning to reduce risk exposure time
%    - Arbitration logic to select "worst" fault in case of several simultaneous control
%      requests by different functions
%
% The functional safety requirements are derived from the safety goals and safe states,
% taking into account any prior architectural assumptions.
% One functional safety requirement can be valid for several safety goals.
%
% If a safe state cannot be reached by transition within an acceptable time interval,
% an emergency operation must be specified (e.g. if immediately switching off a system
% would not help, it might help to have it function in a degraded state).
%
% If assumptions are made about the reaction of the driver or other people at risk, the
% assumed actions must be specified in the functional safety concept and the controls
% available to the driver to do this must be specified.
%
% Required information:
%    - At least one functional safety requirement for each safety goal, specified by 
%      considering:
%        - Operating modes of the item
%        - Time interval of fault tolerance, i.e. how long before hazardous event occurs
%        - Safe states of the item (from safety goals)
%        - Emergency operation interval
%        - Functional redundancies (fault tolerance)
%
% Reference: DS/ISO 26262-3 section 8.4.2
\begin{description}
\item [1-FR1-D] To prevent a violation of 1-SG1-D and 1-SG2-D in the event of a light emitting device malfunction, two different systems, employing two different light emitting devices, shall be controlled by the brake system monitoring.
This will immediately lead to a degraded, but safe, state in case of a light emitting device malfunction.
Continuing to drive after transitioning to the degraded state is considered emergency operation, and the malfunction must be corrected as soon as possible.

\item [1-FR2-D] To prevent a violation of 1-SG2-D in the event of a brake light control system malfunction, a safe state shall be maintained by implementing the brake light control system as a normally closed system, so that the brake light will shine when no control signal is detected.
Continuing to drive while no control signal is detected is considered emergency operation, and the malfunction must be corrected as soon as possible.

\item [1-FR3-D] The brake light system and the driver can not detect whether a malfunction has occurred, so to prevent the violation of 1-SG3-D, a bystander will have to alert the driver, who must then bring the car to a stop as soon as possible.
This can be done by e.g. signalling the driver via flags.

\item [1-FR4-D] To prevent a violation of 1-SG4-D in the event of a brake pedal position sensor malfunction, two different systems, employing two different brake pedal position sensors, shall be employed by the brake system monitoring to determine the position of the brake pedal.
If the two systems disagree, the tractive system must be disabled and the driver must be alerted.

\item [1-FR5-D] To prevent a violation of 1-SG4-D in the event of a brake pedal position sensor giving an implausible measurement, the brake pedal position sensors shall be confined to a limited range of operation, and values outside of this range shall send a signal to High Voltage Control to disable the tractive system.
\end{description}

\subsection{Allocation of functional safety requirements}
% Each functional safety requirement shall be allocated to the elements of the prior
% architectural assumptions.
%
% When allocating, the ASIL is inherited from the associated safety goal.
% If several functional safety requirements are allocated to the same architectural
% element, the highest ASIL is inherited unless it can be argued that the safety
% requirements are independent in the architecture.
% 
% If an item consists of more than one system, the functional safety requirements
% for the individual systems and their interfaces shall be specified.
%
% If functional safety requirements are implemented by external measures, the safety
% requirements that are implemented shall be derived and communicated.
% Additionally, the safety requirements of the interfaces with external measures
% shall be specified.
%
% Required information:
%    - Allocation of each functional safety requirement to elements of the architecture
%    - ASIL for development of each functional safety requirement
%    - Functional safety requirements of interfaces to external measures/internal systems
%
% Reference: DS/ISO 26262-3 section 8.4.3
The brake pedal position sensor shall implement the following functional safety requirements:
\begin{itemize}
\item 1-FR4-D.
\end{itemize}
The brake light shall implement the following functional safety requirements:
\begin{itemize}
\item 1-FR1-D
\item 1-FR2-D
\end{itemize}
The team and other bystanders shall implement the following functional safety requirements:
\begin{itemize}
\item 1-FR3-D.
\end{itemize}
