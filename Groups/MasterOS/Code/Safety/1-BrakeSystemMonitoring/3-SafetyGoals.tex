\section{Safety goals}
% Reference: DS/ISO 26262-3 section 7.4.4.3 -- 7.4.4.6
%
% Pre-requisites:
%    - Item definition
%    - Hazard analysis and risk assessment
%
% The item is evaluated based ONLY on the item definition, i.e. WITHOUT considering
% internal safety mechanisms that are already implemented or are planned.
% Safety mechanisms that are intended to be implemented are instead part of the
% functional safety concept.
% 
% Reference: DS/ISO 26262-3 section 7.4.1

\subsection{Pre-existing information}
% This section is for further supporting information such as
%    - Change impact analysis if applicable (only for existing items being changed)
%    - Relevant information on other independent items
%
% Reference: DS/ISO 26262-3 section 7.3.2
The brake system monitoring is a new item, so there will be no change impact.

% Every hazardous event with an ASIL from the hazard analysis shall have a safety goal
% defined to prevent the malfunction leading to the event.
% If several hazardous events cause very similar safety goals, the safety goals may be
% combined into a single safety goal.
%
% Safety goals are top-level safety requirements for the item. They are not expressed
% in terms of technological solutions but in terms of functional objectives cf. the 
% item definition.
% The safety goals will lead to functional safety requirements required to avoid risk
% of hazardous events later on.
%
% Each safety goal shall be assigned the ASIL of the corresponding hazardous event.
% If multiple safety goals are combined, the highest ASIL must be assigned.
%
% If a safety goal can be achieved by transitioning to or maintaining one or more
% safe states, then those safe states shall be specified.
% A safe state can be e.g. that the item is turned off or that the vehicle is stationary.
%
% Required information:
%    - Safety goals for the hazardous event as described above
%    - For each safety goal:
%        - Safe states to transition to/maintain
%    - Each safety goal must be:
%        - Unambigous
%        - Comprehensible
%        - Atomic
%        - Internally consistent
%        - Feasible
%        - Verifiable
%        - Uniquely identifiable by some ID
%    - The set of safety goals must be:
%        - Complete
%        - Externally consistent
%        - Without duplication of information
%        - Maintainable
%
% References:
%    - Safety goals: DS/ISO 26262-3 section 7.4.4.3 -- 7.4.4.6
%    - Specification of safety requirements: DS/ISO 26262-8 section 6.4
%
% <<< Repeat this section for each hazardous event >>>
\subsection{1-OS4-HZ2 - ASIL D}
This hazardous event is caused by the brake light not shining when braking.
To prevent this malfunction, we set the following safety goals:
\begin{description}
\item [1-SG1-D] The brake light must transition to the shining state when the brake pedal position sensor indicates that the brake pedal is pressed, and it must transition to the non-shining state when the brake pedal position sensor indicates that the brake pedal is released.
\item [1-SG2-D] Since no hazardous event assigned an ASIL involves the brake light shining unintentionally, the brake light must shine when no control signal is detected.
\item [1-SG3-D] When a malfunction is detected, the car must be brought to a stop so the malfunction can be diagnosed and corrected.
\item [1-SG4-D] The brake pedal position sensor must indicate that the brake pedal is pressed only if it is actually pressed, and it must indicate that the brake pedal is released only if it is actually released.
\end{description}

\subsection{1-OS8-HZ2 - ASIL B}
This hazardous event is caused by the brake light not shining when braking.
As such, it is covered by the following safety goals: 1-SG1-D, 1-SG2-D, 1-SG3-D, and 1-SG4-D.

\subsection{1-OS10-HZ2 - ASIL B}
This hazardous event is caused by the brake light not shining when braking.
As such, it is covered by the following safety goals: 1-SG1-D, 1-SG2-D, 1-SG3-D, and 1-SG4-D.

\subsection{1-OS5-HZ2 - ASIL A}
This hazardous event is caused by the brake light not shining when braking.
As such, it is covered by the following safety goals: 1-SG1-D, 1-SG2-D, 1-SG3-D, and 1-SG4-D.

\subsection{1-OS7-HZ2 - ASIL A}
This hazardous event is caused by the brake light not shining when braking.
As such, it is covered by the following safety goals: 1-SG1-D, 1-SG2-D, 1-SG3-D, and 1-SG4-D.
