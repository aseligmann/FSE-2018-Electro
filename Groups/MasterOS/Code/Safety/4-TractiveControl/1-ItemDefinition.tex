\section{Item definition}
% Reference: DS/ISO 26262-3 section 5

\subsection{Pre-existing information}
% This section is for further supporting information such as a product idea,
% a project sketch, relevant patents, the results of pre-trials, the documentation
% from predecessor items, relevant information on other independent items.
% Reference: DS/ISO 26262-3 section 5.3.21
The tractice control system consists of a pedal, a computation system and steering system.
Furthermore extern items like motor controllers, motors and motor encoders are used.

\subsection{Functional requirements}
% A functional requirement is a requirement for what the item is supposed
% to accomplish.
% In general, functional requirements can be expressed in the form:
%     The item must do <requirement>
% Functional requirements are best specified by input/output listings,
% a mathemathical model/function or similar behaviour descriptions.
%
% Required information:
%   - Purpose of the item
%   - Functionality of the item, including operating modes and states of the item
%   - Assumptions on expected behaviour of the item
%
% Reference: DS/ISO 26262-3 section 5.4.1
The tractive system has a red light that blinks in a frequency of 2 and 5 HZ when the tractive system is active.
When the tractive system is deactivated a green light will be illuminated.

The purpose of the speed pedal is to generate a speed reference based on pedal position.

Whenever the pedal is not pressed the motors should not contribute to accelerating the vehicle.
Whenever the pedal is pressed the motors should contribute to accelerating the vehicle.
The further the pedal is pressed the more the motors should contribute to accelerating the vehicle.

The pedal system must always operate when the tractive system is active.

\subsection{Non-functional requirements}
% A non-functional requirement is a requirement for how the item should be.
% In general, non-functional requirements can be expressed in the form:
%     The item must be <requirement>
% Non-functional requirements are overall properties of the item such as quality
% requirements and constraints.
%
% Required information:
%    - Operational constraints such as prescribed behaviours for implementation
%    - FSE rule requirements
%
% Recommended information:
%    - Environmental constraints such as temperature and humidity
%    - Behaviour achieved by similar items, if any
%
% Reference: DS/ISO 26262-3 section 5.4.1
The tractive system must not have a voltage greater than 600V or exceed a power greater than 80KW.
Additionally the system must have power measuring points.

The computation system must not exceed temperatures above 80°C and should not shutdown in harsh environmental conditions.

The speed pedal and steering wheel system must be resistant to the environmental conditions inside the cockpit.


\subsection{Boundary and interfaces}
% Required information:
%    - Elements of the item
%    - Assumptions about item's behaviour on other items
%    - Interactions with other items
%    - Functionality required by other items
%    - Functionality required from other items
%
% Recommended information:
%    - Distribution of functions among the involved systems
%    - Operating scenarios which impact the functionality of the item
%
% Reference: DS/ISO 26262-3 section 5.4.2

The speed pedal system must output the speed pedal position to the computation item and the high voltage control item.
The steering wheel system must output the steering wheel position to computation item.
The computation system must output the motor voltage references to the motor controllers.

